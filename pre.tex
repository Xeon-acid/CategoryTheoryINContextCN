% XeLaTeX 基本设置
\usepackage{xeCJK}
\usepackage{fontspec}
% \setCJKmainfont{SimSun} % 可根据需要更换为其他中文字体
% \setmainfont{Times New Roman}

% 其他常用宏包
% \usepackage{unicode-math}
\usepackage{amsmath, amssymb, amsthm}
\usepackage{geometry}
\usepackage{hyperref}
\hypersetup{
    colorlinks=true,
    linkcolor=blue,
    citecolor=blue,
    urlcolor=blue,
    pdfborder={0 0 0}
}
\usepackage{graphicx}
\usepackage{fancyhdr}
\usepackage{titlesec}
\usepackage{tocloft}
\usepackage{pgfplots}
\usepackage{indentfirst}
\usepackage{tikz}
\usetikzlibrary{cd}
\usepackage{pgffor}
\usepackage{amsfonts}
\usepackage{newtxmath}
\usepackage{footnote}
\usepackage{enumitem}
\usepackage{float}
\RequirePackage{etoolbox}
% \usepackage[mtpbbi]{mtpro2}
% \usepackage[bbgreekl]{mathbbol}
% \setmathfontface\mathbbgreek{Some Outline Greek Font}[Scale=MatchUppercase]


% 页面设置
\geometry{a4paper, left=3cm, right=2.5cm, top=3cm, bottom=3cm}

% 页眉页脚
\pagestyle{fancy}
\fancyhead{}
\fancyfoot{}
\fancyhead[LE,RO]{\thepage}
\fancyhead[RE]{\leftmark}
\fancyhead[LO]{\rightmark}

% 章节标题格式
\titleformat{\chapter}[hang]{\Huge\bfseries}{第\,\thechapter\,章}{2em}{}
\titleformat{\section}{\Large\bfseries}{\thesection}{1em}{}
\titleformat{\subsection}{\large\bfseries}{\thesubsection}{1em}{}

% 目录设置
\renewcommand{\contentsname}{目录}
\setcounter{tocdepth}{2}

% 定理环境
\theoremstyle{definition}
\newtheorem{definition}{定义}[chapter]
\newtheorem{theorem}{定理}[chapter]

\newtheorem{proposition}{命题}[chapter]
\newtheorem{corollary}{推论}[chapter]
% \theoremstyle{remark}
\newtheorem{remark}{注解}[chapter]
% \theoremstyle{definition}
\newtheorem{example}{例}[chapter]
\theoremstyle{plain}
\newtheorem{lemma}{引理}[chapter]
\newcommand{\squarebrace}[1]{「#1」}

\RequirePackage{needspace}
% 章节前言
\newenvironment{says}[3]
{
    \hfill%
    \begin{minipage}{.618\linewidth}
        \begin{flushleft}
            #1%
        \end{flushleft}
        \hrule
        \begin{flushright}
            #2\cite{#3}%
        \end{flushright}
    \end{minipage}%
    %
    \par\vspace{2em}
}{
}

% 如果你想用 \says{内容}{作者}{引用},可以这样定义:
% \newcommand{\says}[3]{%
%   \begin{flushleft}#1\end{flushleft}%
%   \hrule
%   \begin{flushright}#2\cite{#3}\end{flushright}%
% }

% \newcommand{\myref}[1]{\S\arabic{chapter}.\arabic{section}~\ref{#1}}
% \usepackage{refcount}

\newcommand{\chptref}[1]{\S\,\ref*{#1}}
% \newcommand{\eqref}[1]{(\ref*{#1})}

% \RequirePackage{etoolbox}
    % 方便地引入数学算子
% \RequirePackage{foreach}

% \def\mylist{1,2,3,0}
%  \foreach \x in \mylist {[\x]}

% MARK: 定义运算符列表
\def\DclrMthOprLst{Ext,Hom,Mor,Ob,End,Aut}
\DeclareMathOperator{\vP}{\Pi}

\typeout{DclrMthOprLst defined}
\foreach \x in \DclrMthOprLst {%
    \expandafter\global\expandafter\edef\csname\x\endcsname{%
    \noexpand\mathop{\kern0pt \fam0 \x}%
}
    % \typeout{\x\space defined}%
}
% \typeout{\Ext defined}
% \DeclareMathOperator\AAA{definition}

\usepackage{relsize}

\renewcommand{\theequation}{\arabic{chapter}.\arabic{section}.\arabic{definition}}

\renewcommand{\thedefinition}{\arabic{chapter}.\arabic{section}.\arabic{definition}}
\renewcommand{\theremark}{\arabic{chapter}.\arabic{section}.\arabic{definition}}
\renewcommand{\theexample}{\arabic{chapter}.\arabic{section}.\arabic{definition}}
\renewcommand{\thelemma}{\arabic{chapter}.\arabic{section}.\arabic{definition}}

\AtBeginEnvironment{example}{\protect\refstepcounter{definition}}
\AtBeginEnvironment{remark}{\protect\refstepcounter{definition}}
\AtBeginEnvironment{lemma}{\protect\refstepcounter{definition}}
\AtBeginEnvironment{equation}{\protect\refstepcounter{definition}}

\pretocmd{\section}{\setcounter{definition}{0}}




% add hook front of \section, hook is : \clearpage
\let\oldsection\section
\renewcommand{\section}{\clearpage\oldsection}

% \def\inenvefootnotemark{%
%     \refstepcounter{footnote}\footnotemark[\arabic{footnote}]\expandafter\let\csname inEnvFootnoteCounter\arabic{footnote}\csname{\arabic{footnote}}
% }

% \makeatletter

% \newcounter{inenv@markindex}
% \newcommand{\inenv@markqueue}{}

% \newcommand{\inenvefootnotemark}{%
%   \refstepcounter{footnote}%
%   \footnotemark[\number\value{footnote}]%
%   \ifx\inenv@markqueue\@empty
%     \xdef\inenv@markqueue{\number\value{footnote}}%
%   \else
%     \xdef\inenv@markqueue{\inenv@markqueue,\number\value{footnote}}%
%   \fi
% }

% \newcommand{\inenvfootnotetext}[1]{%
%   \inenv@getnextmark
%   \ifnum\inenv@currentmark>0
%     \footnotetext[\inenv@currentmark]{#1}%
%   \else
%     \PackageWarning{inenvfootnote}{No more marks available!}%
%   \fi
% }

% \newcommand{\inenv@getnextmark}{%
%   \@ifundefined{inenv@mark@parsed}{\inenv@parsemarks}{}%
%   \stepcounter{inenv@markindex}%
%   \@ifundefined{inenv@mark@\roman{inenv@markindex}}
%     {\def\inenv@currentmark{0}}
%     {\edef\inenv@currentmark{\csname inenv@mark@\roman{inenv@markindex}\endcsname}}%
% }

% \newcommand{\inenv@parsemarks}{%
%   \gdef\inenv@mark@parsed{}%
%   \@tempcnta=0
%   \@for\inenv@temp:=\inenv@markqueue\do{%
%     \advance\@tempcnta by 1
%     \expandafter\xdef\csname inenv@mark@\romannumeral\@tempcnta\endcsname{\inenv@temp}%
%   }%
%   \setcounter{inenv@markindex}{0}%
% }

% \makeatother


% \makeatletter
% % \IgnoreSpacesOn
% \tlNew \gFootNoteTl
% \intNew \gFootNoteInt
% \prgNewFunction \footNote {m}
%     {%
%         \tlPutRight \gFootNoteTl%
%         {%
%             \stepcounter{footnote}%
%             \footnotetext{#1}%
%         }%
%         \prgReturn {\footnotemark{}}%
%     }
% \AddToHook{env/tblr/before}{%
%     \intSetEq \gFootNoteInt \c@footnote%
%     \tlClear \gFootNoteTl%
%     }
% \AddToHook{env/tblr/after}{%
%     \intSetEq \c@footnote \gFootNoteInt%
%     \tlUse \gFootNoteTl%
%     }
% % \IgnoreSpacesOff
% \makeatother

\newcommand{\cat}[1]{\mathsf{#1}}

\let\alias\let


\newcommand\mathbbpdf[1]{\special{pdf:literal 1 Tr 0.3 w}#1\special{pdf:literal 0 Tr 0 w}}

\newcounter{exercise}

\newcommand{\Exercises}{%
    \setcounter{exercise}{0}%
    \vskip0ex\kern4ex{\textbf{习题.}}%
}

\newcommand{\exercise}[1]{\refstepcounter{exercise}\vspace{.5em}\par\noindent{}习题\kern.5ex\theexercise.\kern.5em#1}

\def\theexercise{%
    \arabic{chapter}.\arabic{section}.\roman{exercise}
}

\setCJKmainfont{SourceHanSerifSC}[
    Path = C:/Users/angel/AppData/Local/Microsoft/Windows/Fonts/,
    Extension = .otf,
    UprightFont = *-Regular,
    BoldFont = *-Bold,
    ItalicFont = SIMKAI,
    BoldItalicFont = SIMKAI,
    AutoFakeBold = true,
    BoldFeatures = {FakeBold=1.5},
    ItalicFeatures = {
        Path = C:/Windows/Fonts/,
        Extension = .ttf,
        % FakeSlant=0.2
    },
    BoldItalicFeatures = {
        Path = C:/Windows/Fonts/,
        Extension = .ttf,
        FakeBold=1.8  % 可以为粗斜体设置不同的粗体强度
    }
]
\setmainfont{CMU Serif}